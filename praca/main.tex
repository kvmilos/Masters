\documentclass[magisterska,en]{pracamgr_Kogni}

\usepackage{graphicx}
\usepackage{setspace}
\usepackage{hyperref}
\usepackage{subfiles}
\usepackage{chngcntr}
\usepackage{enumitem}
\usepackage{tikz-dependency} 
\usepackage{float}
\usepackage[section]{placeins}
\counterwithout{figure}{chapter}
\counterwithout{table}{chapter} % if you want the same for tables


% ---- Bibliography (biblatex + biber) ----
% giveninits=true works with uniquename=init to avoid conflicts
\usepackage[backend=biber,style=authoryear,sorting=nyt,maxnames=6,giveninits=true,uniquename=init]{biblatex}
\addbibresource{proposed.bib}
\usepackage{csquotes}


\hypersetup{
  colorlinks=true,
  linkcolor=black,
  citecolor=black,
  urlcolor=black,
}

\onehalfspacing
\counterwithout{footnote}{chapter}

\autor{Kamil Tomaszek}{432044}
\title{Middle Polish Dependency Treebank in Universal Dependencies format: Design, Implementation, and Analysis}
\kierunek{Cognitive Science}
\opiekun{Dr. Alina Wróblewska\\
Institute of Computer Science\\
Polish Academy of Sciences\\
\bfseries Dr. Grzegorz Krajewski\\
University of Warsaw}
\date{September 2025}
\keywords{Middle Polish, dependency trees, treebank conversion, Universal Dependencies}
\keywordspl{język średniopolski, drzewa zależnościowe, konwersja korpusu, Universal Dependencies}
\tytulang{Middle Polish Dependency Treebank in Universal Dependencies format: Design, Implementation, and Analysis}

\begin{document}
\pagenumbering{gobble}% hide page numbers for front matter (title, abstracts, ToC)
\maketitle

\titlepl{Średniopolski Bank Drzew Zależnościowych w formacie Universal Dependencies: projekt, implementacja i analiza}
\begin{abstract}
This thesis presents a rule-based approach to converting the Middle Polish Dependency Treebank (MPDT), annotated in a Polish-specific scheme, into the Universal Dependencies (UD) format. After introducing the project motivation, data sources, and target standard, the thesis outlines general design assumptions behind the conversion, the mapping strategy, and the validation workflow. It reports overall outcomes of the conversion and sketches applications and extensions, including releasing MPDT-UD and implications for research in historical language processing within cognitive science.
\end{abstract}

\begin{abstractpl}
Praca przedstawia podejście regułowe do konwersji Średniopolskiego Banku Drzew Zależnościowych (MPDT), anotowanego w polskim schemacie, do formatu Universal Dependencies (UD). Po krótkim omówieniu motywacji, danych i standardu docelowego zaprezentowano ogólne założenia projektu, strategię odwzorowań oraz schemat walidacji. Przedstawiono ogólne wyniki konwersji oraz możliwe zastosowania i kierunki rozwoju, w tym udostępnienie MPDT-UD i znaczenie dla badań nad przetwarzaniem języka historycznego w kognitywistyce.
\end{abstractpl}

\tableofcontents

% --- CHAPTER 1 ---
\cleardoublepage
\pagenumbering{arabic}
\setcounter{page}{5}
\chapter{Introduction}
\subfile{sections/intro}

% --- CHAPTER 2 ---
\chapter{Background}
\subfile{sections/background}

% --- CHAPTER 3 ---
\chapter{Linguistic Features of Middle Polish}
\subfile{sections/linguistics}

% --- CHAPTER 4 ---
\chapter{Conversion Design and Implementation}
\subfile{sections/approach}

% --- CHAPTER 5 ---
\chapter{Validation and Outcomes}
\subfile{sections/validation}

% --- CHAPTER 6 ---
\chapter{Applications and Cognitive Science Perspective}
\subfile{sections/applications}

\printbibliography

\end{document}
