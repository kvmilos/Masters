%!TEX root = ../main.tex
\documentclass[../main.tex]{subfiles}
\begin{document}

\section{Motivation}

Natural-language tools and comparative treebank research have standardized around~Universal Dependencies (UD),
which enables typologically informed analyses and cross-lingual transfer \parencite{nivre-etal-2020-universal}.
For~17th–18th-century Polish, however, key resources remain outside~UD:
Middle Polish texts in~KorBa \parencite{Gruszczynski2022KorBa} and the emerging Middle Polish Dependency Treebank
(MPDT) are annotated in~a~Polish-specific scheme \parencite{Wieczorek2025TowardsMPDT}. This limits their
interoperability with~UD-based tools and does not allow for~straightforward comparative studies with~other languages.

From~an engineering perspective, a~faithful, auditable conversion is non-trivial:
historical orthography, abbreviations (\texttt{brev}), clitic mobility (\textit{by}, \textit{że}),
numeral complexes, and multiword conjunctions/prepositions interact with~head rules and label inventories.
Prior conversion experience for~contemporary Polish (PDB~$\rightarrow$~PDB-UD; NKJP1M~$\rightarrow$~NKJP1M-UD) offers valuable guidance
\parencite{wroblewska-2018-extended, wroblewska-2020-towards},
yet historical data introduce additional phenomena that require explicit, rule-based handling and transparent traceability.

As~\textcite{Wieczorek2025TowardsMPDT} notes, MPDT’s current PDB-consistent format is~well-suited to~comparative
studies with~contemporary Polish syntax; at~the same time, she highlights the advantages of~moving to~UD
for~cross-linguistic comparability, wider intelligibility, and representational options such as~enhanced dependencies
for~shared dependents and shared governors in~coordination—even if~some information may be lost in~translation.
This thesis operationalizes that rationale by~delivering a~documented, UD-oriented conversion for~MPDT and preparing
an initial MPDT-UD subset suitable for~validation and downstream use.


\section{Objectives}

The thesis pursues the following goals:

\begin{enumerate}[label=\textbf{(O\arabic*)}, leftmargin=2em]
  \item \textbf{Design a~UD-oriented conversion strategy for~MPDT.}
        Specify mapping principles that respect Middle Polish specifics while aligning with~UD guidelines.
  \item \textbf{Implement an auditable conversion pipeline.}
        Provide modular components for~morphosyntax mapping and dependency restructuring, with~token-level logging.
  \item \textbf{Ensure UD conformance and evaluability.}
        Produce output that passes the official UD validator (on all levels) and supports downstream analysis.
  \item \textbf{Document decisions.}
        Record non-obvious mapping choices and edge-case policies to~enable maintenance and reuse.
\end{enumerate}


\section{Contributions}

This project delivers concrete, reusable artifacts:

\begin{enumerate}[label=\textbf{(C\arabic*)}, leftmargin=2em]
  \item \textbf{A~rule-based MPDT~$\rightarrow$~UD converter.}
        A~modular pipeline with~fine-grained logging, selectively adapting ideas from~PDB$\rightarrow$UD while
        targeting Middle Polish phenomena. The code will be released in~a~public repository under~an open-source
        license, together with~this paper, which documents the design and implementation.
  \item \textbf{An initial public release of~MPDT-UD.}
        A~subset of~MPDT (2018 sentences at~the time of~writing) converted automatically and validated with~the
        official UD validator.
\end{enumerate}

\noindent The intended users include historical linguists needing UD-compatible data and~NLP
practitioners interested in~diachronic Polish or~cross-lingual experiments.


\section{Structure of the Document}

\begin{itemize}[leftmargin=1.5em]
  \item \textbf{Chapter~2: Background.} Presents the foundational concepts and resources, including dependency grammar, Universal Dependencies, KorBa and MPDT.
  \item The remaining chapters will be incorporated here as they are finalized.
\end{itemize}

\end{document}
