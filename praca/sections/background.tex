\documentclass[../main.tex]{subfiles}
\begin{document}

\section{Dependency Grammar}

\section{Universal Dependencies}

\section{Resources}
\subsection{KorBa}

KorBa is a~13.5-million-token corpus of~Polish texts from~1601–1772, compiled from~over seven hundred sources
and annotated morphosyntactically (lemmas, POS, features). It is searchable via~MTAS (Multi Tier Annotation Search),
and provides parallel transliteration/transcription layers, structural and language markup, and rich metadata
(period, region, text type, genre) that enable stratified analyses \parencite{Gruszczynski2022KorBa}.

In~this thesis, KorBa supplies the textual and morphosyntactic substrate.

\begin{figure}[H]
  \centering
  \includegraphics[width=\linewidth]{figs/korba_map.pdf}
  \caption{Geographical distribution of~texts in~the corpus displayed on~the map of~the Commonwealth after~the Union of~Lublin of~1569}
  \label{fig:korba-map}
\end{figure}

\begin{figure}[H]
  \centering
  \includegraphics[width=\linewidth]{figs/korba_sources.pdf}
  \caption{Types of~texts}
  \label{fig:korba-sources}
\end{figure}

\noindent\emph{Source for~Figures~\ref{fig:korba-map}--\ref{fig:korba-sources}:} \textcite{Gruszczynski2022KorBa}, CC~BY~4.0.

\subsection{MPDT}

Middle Polish Dependency Treebank is a~syntactically annotated subset of~KorBa: it adds a~dependency layer on~top of~KorBa’s tokenization, lemmas,
POS, and FEATS for~selected Middle Polish texts \parencite{Wieczorek2025TowardsMPDT}.
The annotation scheme is compatible with~PDB conventions, and the resource is under active development (not publicly released at~the time of~writing).

In~this thesis, the converter consumes MPDT data—i.e., KorBa’s tokenization, lemmata, POS and features
\emph{together with} the MPDT dependency layer—and transforms them to~UD (CoNLL-U).
Only those KorBa segments that belong to~MPDT are converted, since a~dependency layer is a~prerequisite for~UD conversion.

\subsection{PDB / PDB-UD (analogy and rule reuse)}

Polish Dependency Treebank and its UD counterpart (PDB-UD) are not data sources for~this work, but they serve as~useful analogies.
Where appropriate, we~reuse or adapt established Polish-specific conversion patterns (e.g., treatment of~function words,
coordination, and numerals) described for~PDB~$\rightarrow$~UD, tailoring them to~Middle Polish phenomena
\parencite{wroblewska-2020-towards}.

\end{document}
