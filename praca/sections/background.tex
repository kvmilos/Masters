\documentclass[../main.tex]{subfiles}
\begin{document}

\section{Dependency Grammar}

Dependency grammar is a~theory of~syntactic structure organized around~asymmetric
head–dependent relations. A~\emph{dependency} links two lexical items: a~\emph{head} that selects and constrains, and
a~\emph{dependent} that is licensed by~the head. Sentences are modeled as~directed trees whose nodes correspond to~tokens and whose
edges encode these head–dependent links. The tree has a~single \emph{root} (a node with no governor), and every other node is
reachable from~it along~directed edges. In~addition to~purely structural links. We~use dependency grammar here in~a~strictly
morphosyntactic sense, leaving semantic or prosodic dependency representations aside.

The example dependency trees below illustrates the notation used in~this thesis. Edge labels are function-like and purely illustrative.

\medskip
\noindent\textbf{Example tree}
\begin{dependency}[theme=simple, baseline=6.8em, label style={font=\footnotesize}]
  \begin{deptext}[column sep=1.8em]
    Chłopiec \& siedzi \& na \& łóżku \& . \\
    boy \& sits \& on \& bed \& . \\
     \& \& \& \& \\
    NOUN \& VERB \& ADP \& NOUN \& PUNCT \\
  \end{deptext}
  \deproot[edge style=dotted]{2}{root}
  \depedge{2}{1}{subj}
  \depedge{2}{3}{comp}
  \depedge{3}{4}{comp}
  \depedge{2}{5}{punct}
\end{dependency}

\medskip
Dependency formalisms differ on~certain design choices (e.g., whether adpositions are heads or dependents inside adpositional phrases;
how to~encode coordination; whether and how to~mark valency vs.~modification). One universal standard is~Universal Dependencies.



\section{Universal Dependencies}

Universal Dependencies (further references as UD) is a~cross-linguistic annotation framework designed to~harmonize
morphosyntactic and syntactic representations across languages within a~dependency-based,
lexicalist model \parencite{nivre-etal-2020-universal}. It is widely adopted in~NLP and linguistic typology, and
serves as~the target formalism for~the conversion presented in this thesis.
The scheme provides three aligned layers:

\begin{enumerate}[leftmargin=2em]
  \item \textbf{Tokenization.} UD defines dependencies between \emph{syntactic words}.
        To handle orthographic contractions or clitic clusters, it uses
        \emph{multiword tokens}, ensuring a~faithful word-level analysis.
        Conversely, several orthographic tokens may be combined into one syntactic word
        in~well-motivated, language-specific cases.
  \item \textbf{Morphology.} Each token is associated with a~\texttt{LEMMA},
        a~universal POS tag (\texttt{UPOS} from a~fixed 17-tag set),
        and a~bundle of~\texttt{FEATS}.
        UD~v2 standardized features and values across languages
        and clarified tag boundaries, e.g.~extending \texttt{AUX} to~copulas and
        tense–aspect–mood particles while narrowing \texttt{PART}.
  \item \textbf{Syntax.} The syntactic layer is a~single-rooted tree with
        universal dependency relations such as \texttt{nsubj}, \texttt{obj}, \texttt{obl},
        \texttt{amod}, \texttt{case}, \texttt{cc}, and \texttt{conj}.
        UD~v2 unified earlier divergences by, for example, replacing \texttt{dobj} with
        \texttt{obj}, reserving \texttt{obl} for predicate-level obliques (distinct
        from nominal \texttt{nmod}), and constraining the use of~\texttt{cop} to
        pure grammatical linkers.
\end{enumerate}

In~addition to~the \emph{basic} representation, UD also defines an \emph{enhanced} graph
that adds extra arcs (and occasionally null nodes) to capture phenomena such as
shared dependents in~coordination, control and raising, relativization, and ellipsis.

\paragraph{Format:}
UD uses the CoNLL-U format, a~ten-column tabular specification with the fields:
\begin{list}{}{
  \setlength{\leftmargin}{2em}
  \setlength{\itemsep}{0.6\itemsep}
}
\item[-] \texttt{ID} - a~token index (or range for multiword tokens);
\item[-] \texttt{FORM} - the surface form;
\item[-] \texttt{LEMMA} - the dictionary form;
\item[-] \texttt{UPOS} - the universal POS tag;
\item[-] \texttt{XPOS} - a~language-specific POS tag;
\item[-] \texttt{FEATS} - a~pipe (|) separated list of~morphological features;
\item[-] \texttt{HEAD} - the index of~the head token (or 0 for~the root);
\item[-] \texttt{DEPREL} - the dependency relation to~the head;
\item[-] \texttt{DEPS} - for~enhanced dependencies;
\item[-] \texttt{MISC} - for~miscellaneous annotations.
\end{list}
\medskip
Here is an example CoNLL-U snippet:
{\scriptsize
\begin{list}{}{
  \setlength{\leftmargin}{-3em}
}
\item
\begin{verbatim}
# sent_id = test-sentence
# text = Chłopiec siedzi na łóżku.
1   Chłopiec   chłopiec  NOUN   subst   Gender=Masc|Number=Sing|Case=Nom                     2   nsubj   _   _
2   siedzi     siedzieć  VERB   fin     Aspect=Imp|Mood=Ind|Tense=Pres|Person=3|Number=Sing  0   root    _   _
3   na         na        ADP    prep    AdpType=Prep|Case=Loc                                4   case    _   _
4   łóżku      łóżko     NOUN   subst   Gender=Neut|Number=Sing|Case=Loc                     2   obl     _   _
5   .          .         PUNCT  interp  PunctType=Peri                                       2   punct   _   _
\end{verbatim}
\end{list}
}
\medskip


\section{Resources}
\subsection{KorBa}

KorBa is a~13.5-million-token corpus of~Polish texts from~1601–1772, compiled from~over seven hundred sources
and annotated morphosyntactically (lemmas, POS, features). It is searchable via~MTAS (Multi Tier Annotation Search),
and provides parallel transliteration/transcription layers, structural and language markup, and rich metadata
(period, region, text type, genre) that enable stratified analyses \parencite{Gruszczynski2022KorBa}.

In~this thesis, KorBa supplies the textual and morphosyntactic substrate.

\begin{figure}[H]
  \centering
  \includegraphics[width=\linewidth]{figs/korba_map.pdf}
  \caption{Geographical distribution of~texts in~the corpus displayed on~the map of~the Commonwealth after~the Union of~Lublin of~1569}
  \label{fig:korba-map}
\end{figure}

\begin{figure}[H]
  \centering
  \includegraphics[width=\linewidth]{figs/korba_sources.pdf}
  \caption{Types of~texts in KorBa}
  \label{fig:korba-sources}
\end{figure}

\noindent\emph{Source for~Figures~\ref{fig:korba-map}--\ref{fig:korba-sources}:} \textcite{Gruszczynski2022KorBa}, CC~BY~4.0.


\subsection{MPDT}

Middle Polish Dependency Treebank is a~syntactically annotated subset of~KorBa: it adds a~dependency layer on~top of~KorBa’s tokenization, lemmas,
POS, and FEATS for~selected Middle Polish texts \parencite{Wieczorek2025TowardsMPDT}.
The annotation scheme is compatible with~PDB conventions, and the resource is under active development (not publicly released at~the time of~writing).

In~this thesis, the converter consumes MPDT data—i.e., KorBa’s tokenization, lemmata, POS and features
\emph{together with} the MPDT dependency layer—and transforms them to~UD (CoNLL-U).
Only those KorBa segments that belong to~MPDT are converted, since a~dependency layer is a~prerequisite for~UD conversion.

\end{document}
