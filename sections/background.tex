%!TEX root = ../main.tex
\documentclass[../main.tex]{subfiles}
\begin{document}

This chapter provides the essential background for understanding the Middle Polish Dependency Treebank conversion to Universal Dependencies. It begins with the theoretical foundations of dependency grammar and its specific Polish manifestation in the Polish Dependency Bank (PDB) scheme (Section \ref{sec:dependency-grammar}). Then it outlines Universal Dependencies as the target framework, highlighting its advantages for cross-linguistic research (Section \ref{sec:universal-dependencies}). Finally, it describes the key resources: KorBa as the source corpus and MPDT as the dependency-annotated dataset that forms the input to our conversion pipeline (Section \ref{sec:resources}).

\section{Dependency Grammar}
\label{sec:dependency-grammar}

Dependency grammar is a~theory of~syntactic structure organized around~asymmetric
governor–dependent relations. A~\emph{dependency} links two lexical items: a~\emph{governor} that selects and constrains a dependent, and
a~\emph{dependent} that is licensed by~the governor. One item can be a governor for multiple dependents, but each dependent has a~single governor. Sentence structures are modeled as~directed trees whose nodes correspond to~tokens and whose
edges encode these governor–dependent links. The tree has a~single \emph{root} (a node with no governor), and every other node is
reachable from~it along~directed edges. In~addition to~purely structural links, dependency grammar is used here in~a~morphosyntactic
sense, focusing on grammatical relations rather than semantic or prosodic dependency representations.

The dependency scheme used in Middle Polish follows the conventions established for the Polish Dependency Bank (PDB), which is adapted specifically for Polish syntax \parencite{Wroblewska2023PDBManual}. The PDB tagset adapts the NKJP tagset \parencite{NKJP_ksiazka}. The PDB annotation scheme uses a comprehensive set of part-of-speech categories and dependency relations designed specifically for Polish morphosyntax.

\paragraph{The PDB tagset includes the following part-of-speech categories:}

\begin{itemize}[leftmargin=2em]
  \item \textbf{Nouns:} \texttt{subst} (noun), \texttt{depr} (depreciative noun)
  \item \textbf{Pronouns:} \texttt{ppron12} (non-third person pronoun), \texttt{ppron3} (third person pronoun), \texttt{siebie} (reflexive pronoun)
  \item \textbf{Adjectives:} \texttt{adj} (adjective), \texttt{adja} (ad-adjectival adjective), \texttt{adjc} (predicative adjective), \texttt{adjp} (prepositional adjective)
  \item \textbf{Verb forms:} \texttt{fin} (finite non-past), \texttt{praet} (past tense), \texttt{imps} (impersonal), \texttt{impt} (imperative), \texttt{inf} (infinitive), \texttt{aglt} (agglutinate of `być'), \texttt{bedzie} (future form of `być'), \texttt{winien} (modal verbs like `winien'), \texttt{pred} (predicative), \texttt{ger} (gerund), \texttt{pcon} (contemporary adverbial participle), \texttt{pant} (anterior adverbial participle), \texttt{pact} (active adjectival participle), \texttt{ppas} (passive adjectival participle)
  \item \textbf{Numerals:} \texttt{num} (cardinal numeral), \texttt{numcomp} (numeral compound)
  \item \textbf{Conjunctions:} \texttt{comp} (subordinating conjunction), \texttt{conj} (coordinating conjunction)
  \item \textbf{Other categories:} \texttt{adv} (adverb), \texttt{brev} (abbreviation), \texttt{dig} (Arabic numeral), \texttt{romandig} (Roman numeral), \texttt{emo} (emoticon), \texttt{fill} (filler), \texttt{frag} (fragment), \texttt{interj} (interjection), \texttt{interp} (punctuation), \texttt{part} (particle), \texttt{prep} (preposition), \texttt{ign} (unrecognized form)
\end{itemize}

\paragraph{The PDB annotation scheme distinguishes several classes of dependency relations:}

\begin{itemize}[leftmargin=2em]
  \item \textbf{Core arguments:} \texttt{subj} (subject), \texttt{obj} (direct object), \texttt{obj\_th} (thematic object), \texttt{comp} (complement), \texttt{comp\_fin} (finite clause complement), \texttt{comp\_inf} (open clause [\textit{infinitive}] complement), \texttt{comp\_ag} (agent complement)
  \item \textbf{Adjuncts and modifiers:} \texttt{adjunct} with semantic subtypes such as \texttt{adjunct\_temp} (temporal), \texttt{adjunct\_loc} (locative), \texttt{adjunct\_dur} (duration), \texttt{adjunct\_caus} (causal), \texttt{adjunct\_mod} (manner), \texttt{adjunct\_emph} (emphatic particle), \texttt{adjunct\_compar} (comparative)
  \item \textbf{Predicate-related:} \texttt{pd} (predicative expression), \texttt{aux} (auxiliary), \texttt{neg} (negation), \texttt{refl} (reflexive)
  \item \textbf{Coordination:} \texttt{conjunct} (coordinated element), \texttt{pre\_coord} (pre-coordinator)
  \item \textbf{Multiword expressions:} \texttt{mwe} (multiword expression), \texttt{ne} (named entity), \texttt{ne\_foreign} (foreign named entity)
  \item \textbf{Special relations:} \texttt{punct} (punctuation), \texttt{vocative} (vocative), \texttt{orphan} (orphaned dependent), \texttt{discourse} (discourse marker), \texttt{parataxis} (parataxis), \texttt{aglt} (mobile inflection), \texttt{imp} (imperative marker), \texttt{cond} (conditional clitic), and \texttt{root} (sentence root)
\end{itemize}

The example dependency trees below illustrate the scheme of a PDB-annotated sentence alongside its UD counterpart, showing the structural differences.

\begin{figure}[H]
  \centering
  \resizebox{\textwidth}{!}{%
    {\scriptsize
    \begin{dependency}[theme=simple, baseline=6.8em, label style={font=\footnotesize}]
      \begin{deptext}[column sep=1.8em, nodes={text width=4.8em, align=center}]
        Chłopiec \& siedzi \& na \& łóżku \& i \& czyta \& . \\
        \textit{boy} \& \textit{sits} \& \textit{on} \& \textit{bed} \& \textit{and} \& \textit{reads} \& \textit{.} \\
         \& \& \& \& \& \& \\
        SUBST \& FIN \& PREP \& SUBST \& CONJ \& FIN \& INTERP \\
      \end{deptext}
      \deproot[edge style=dotted]{5}{root}
      \depedge{5}{2}{conjunct}
      \depedge{5}{1}{subj}
      \depedge{2}{3}{adjunct\_locat}
      \depedge{3}{4}{comp}
      \depedge{5}{6}{conjunct}
      \depedge{5}{7}{punct}
    \end{dependency}%
    }
  }
  \caption{Example dependency tree in the PDB format}
  \label{fig:pdb-example}
\end{figure}

\begin{figure}[H]
  \centering
  \resizebox{\textwidth}{!}{%
    {\scriptsize
    \begin{dependency}[theme=simple, baseline=6.8em, label style={font=\footnotesize}]
      \begin{deptext}[column sep=1.8em, nodes={text width=4.8em, align=center}]
        Chłopiec \& siedzi \& na \& łóżku \& i \& czyta \& . \\
        \textit{boy} \& \textit{sits} \& \textit{on} \& \textit{bed} \& \textit{and} \& \textit{reads} \& \textit{.} \\
         \& \& \& \& \& \& \\
        NOUN \& VERB \& ADP \& NOUN \& CCONJ \& VERB \& PUNCT \\
      \end{deptext}
      \deproot[edge style=dotted]{2}{root}
      \depedge{2}{1}{nsubj}
      \depedge{4}{3}{case}
      \depedge{2}{4}{obl}
      \depedge{6}{5}{cc}
      \depedge{2}{6}{conj}
      \depedge{2}{7}{punct}
    \end{dependency}%
    }
  }
  \caption{A dependency tree of the same sentence in UD format}
  \label{fig:ud-example}
\end{figure}


Dependency formalisms differ on~certain design choices (e.g., whether adpositions are heads or dependents inside adpositional phrases;
how to~encode coordination; whether and how to~mark valency vs.~modification). The PDB scheme takes specific positions on these issues, treating prepositions as heads (note the \textit{on}$\rightarrow$\textit{bed} relation in \autoref{fig:pdb-example}), using a coordination-centric approach where conjunctions govern coordinated elements (\textit{i} being the \texttt{root} of both \textit{siedzi} and \textit{czyta} in \autoref{fig:pdb-example}).

\section{Universal Dependencies}
\label{sec:universal-dependencies}

Universal Dependencies (hereafter UD) is a~cross-linguistic annotation framework designed to~harmonize
morphosyntactic and syntactic representations across languages within a~dependency-based,
lexicalist model \parencite{nivre-etal-2020-universal, de-marneffe-manning-nivre-zeman-2021}. UD serves as both a theoretical framework and a practical collection of treebanks—currently the largest repository of over 200 treebanks for more than 150 languages.\footnote{Universal Dependencies, \url{https://universaldependencies.org}, accessed 2025-10-10.} It is widely adopted in~NLP and linguistic typology studies, and is maintained by an open community with regular releases.

\paragraph{Annotation scheme:}
The scheme provides three aligned layers for sentence-level annotation:

\begin{enumerate}[leftmargin=2em]
  \item \textbf{Tokenization.} UD defines dependencies between \emph{syntactic words}.
        To handle orthographic contractions or clitic clusters, it uses
        \emph{multiword tokens}, ensuring a~faithful word-level analysis.
        A multiword token is a single orthographic unit that is split into multiple syntactic words, each receiving its own morphological analysis and syntactic function.

        For example, Middle Polish \textit{kiedym} `when I' is annotated as:
\begin{verbatim}
14-15   kiedym     _       _     ...
14      kiedy      kiedy   ADV   ...
15      m          być     AUX   ...
\end{verbatim}
        Here, the single orthographic token \textit{kiedym} (ID 14-15) splits into two syntactic words: \textit{kiedy} `when' (ID 14) and \textit{m} (mobile inflection form of `I am', ID 15).

        Similarly, \textit{jeszcześ} `still you are' becomes:
\begin{verbatim}
7-8   jeszcześ   _         _      ...
7     jeszcze    jeszcze   PART   ...
8     ś          być       AUX    ...
\end{verbatim}

  \item \textbf{Morphology.} Each syntactic word is associated with a~\texttt{LEMMA}, a~universal part-of-speech tag (hereafter part-of-speech tag$=$\texttt{POS}; universal part-of-speech tag$=$\texttt{UPOS}) from a~fixed 17-tag set, and a~bundle of~\texttt{FEATS} (morphological features). The \texttt{UPOS} tags cover open-class words (adjectives \texttt{ADJ}, adverbs \texttt{ADV}, interjections \texttt{INTJ}, nouns \texttt{NOUN}, proper nouns \texttt{PROPN}, verbs \texttt{VERB}), closed-class words (adpositions \texttt{ADP}, auxiliary verbs \texttt{AUX}, coordinating conjunctions \texttt{CCONJ}, determiners \texttt{DET}, numerals \texttt{NUM}, pronouns \texttt{PRON}, particles \texttt{PART}, subordinating conjunctions \texttt{SCONJ}), and other categories (punctuation \texttt{PUNCT}, symbols \texttt{SYM}, other \texttt{X}). UD~v2 standardized features and values across languages and clarified tag boundaries, e.g.,~extending auxiliary verbs to~copulas and tense–aspect–mood particles while narrowing particles. The list of \texttt{UPOS} categories is available on the UD webpage.\footnote{Universal Dependencies POS tags: \url{https://universaldependencies.org/u/pos/index.html}}

  \item \textbf{Syntax.} The syntactic layer is a~single-rooted tree with possible
        37 universal dependency relations organized according to functional and structural categories.
        Sentence structures are modeled as directed trees according to the principles of dependency grammar as described in \ref{sec:dependency-grammar}.
        Relations include:
        \begin{itemize}
         \item core arguments (nominal subject \texttt{nsubj}, direct object \texttt{obj}, indirect object \texttt{iobj}, clausal subject \texttt{csubj}, clausal complement \texttt{ccomp}, open clausal
         complement \texttt{xcomp}),
        \item non-core dependents (oblique \texttt{obl}, dislocated element \texttt{dislocated}, adverbial clause modifier \texttt{advcl}, adverbial modifier \texttt{advmod}, discourse element \texttt{discourse}, auxiliary \texttt{aux}, copula \texttt{cop}, vocative \texttt{vocative}, expletive \texttt{expl}, marker \texttt{mark}),
        \item nominal dependents (nominal modifier \texttt{nmod}, numeral modifier \texttt{nummod}, adjectival modifier \texttt{amod}, determiner \texttt{det}, case marker \texttt{case}, classifier \texttt{clf}, clausal modifier of noun \texttt{acl}, appositional modifier \texttt{appos}),
        \item coordination (conjunct \texttt{conj}, coordinating conjunction \texttt{cc}), 
        \item multiword expressions (fixed \texttt{fixed}, flat \texttt{flat}),
        \item special relations (list element \texttt{list}, parataxis \texttt{parataxis}, orphan \texttt{orphan}, punct \texttt{punct}, root \texttt{root}, overridden disfluency \texttt{reparandum}, relation `goes with' \texttt{goeswith}, other dependent \texttt{dep}).
        \end{itemize}
        The framework also allows language-specific subtypes (e.g., \texttt{nsubj:pass} for passive subjects, \texttt{det:poss} for possessive determiners)
        and defines semi-mandatory subtypes that should be used when the relevant phenomenon exists in the language. A full list of relations and subtypes, along with their descriptions, is available in the UD webpage.\footnote{Universal Dependencies relations list: \url{https://universaldependencies.org/u/dep/index.html}}

\end{enumerate}

In~addition to~the \emph{basic} representation, UD also defines an \emph{enhanced} graph
that adds extra arcs (and occasionally null nodes) to capture phenomena such as
shared dependents in~coordination, control and raising, relativization, and ellipsis.
In \autoref{fig:ud-example}, the basic tree structure is shown; an enhanced representation would add an additional edge to represent the dependent (in this case: the subject) of \textit{czyta} (`reads') as also being
the \textit{Chłopiec} (`boy'), as shown in figure \autoref{fig:ud-enhanced-example}.
\begin{figure}[H]
  \centering
  \resizebox{\textwidth}{!}{%
    {\scriptsize
    \begin{dependency}[theme=simple, baseline=6.8em, label style={font=\footnotesize}]
      \begin{deptext}[column sep=1.8em, nodes={text width=4.8em, align=center}]
        Chłopiec \& siedzi \& na \& łóżku \& i \& czyta \& . \\
        \textit{boy} \& \textit{sits} \& \textit{on} \& \textit{bed} \& \textit{and} \& \textit{reads} \& \textit{.} \\
         \& \& \& \& \& \& \\
        NOUN \& VERB \& ADP \& NOUN \& CCONJ \& VERB \& PUNCT \\
      \end{deptext}
      \deproot[edge style=dotted]{2}{root}
      \depedge{2}{1}{nsubj}
      \depedge{4}{3}{case}
      \depedge{2}{4}{obl}
      \depedge{6}{5}{cc}
      \depedge{2}{6}{conj}
      \depedge{2}{7}{punct}
      \depedge[edge style=dashed]{6}{1}{nsubj}
    \end{dependency}%
    }
  }
  \caption{A dependency tree with enhanced dependencies (dashed lines)}
  \label{fig:ud-enhanced-example}
\end{figure}

\paragraph{Format:}

For practical implementation and data sharing, UD annotations must be encoded in a standardized format. UD uses the CoNLL-U format, a~ten-column tabular specification with the fields:

\begin{list}{}{
  \setlength{\leftmargin}{2em}
  \setlength{\itemsep}{0.6\itemsep}
}
\item[-] \texttt{ID} - a~syntactic word index (or range for multiword tokens);
\item[-] \texttt{FORM} - the surface form;
\item[-] \texttt{LEMMA} - the dictionary form;
\item[-] \texttt{UPOS} - the universal POS tag;
\item[-] \texttt{XPOS} - a~language-specific POS tag;
\item[-] \texttt{FEATS} - a~pipe (|) separated list of~morphological features;
\item[-] \texttt{HEAD} - the index of~the head syntactic word (or 0 for~the root);
\item[-] \texttt{DEPREL} - the dependency relation to~the head;
\item[-] \texttt{DEPS} - for~enhanced dependencies;
\item[-] \texttt{MISC} - for~miscellaneous annotations.
\end{list}
\medskip

Here is a CoNLL-U snippet for the sentence ``Chłopiec siedzi na łóżku i czyta.'', with the enhanced dependencies.

{\scriptsize
\begin{list}{}{
  \setlength{\leftmargin}{-3em}
}
\item
\begin{verbatim}
# sent_id = test-sentence
# text = Chłopiec siedzi na łóżku i czyta.
1   Chłopiec   chłopiec  NOUN   subst   Gender=Masc|Number=Sing|Case=Nom                     2   nsubj   _         _
2   siedzi     siedzieć  VERB   fin     Aspect=Imp|Mood=Ind|Tense=Pres|Person=3|Number=Sing  0   root    _         _
3   na         na        ADP    prep    AdpType=Prep|Case=Loc                                4   case    _         _
4   łóżku      łóżko     NOUN   subst   Gender=Neut|Number=Sing|Case=Loc                     2   obl     _         _
5   i          i         CCONJ  conj    _                                                    2   cc      _         _
6   czyta      czytać    VERB   fin     Aspect=Imp|Mood=Ind|Tense=Pres|Person=3|Number=Sing  2   conj    1:nsubj   _
7   .          .         PUNCT  interp  PunctType=Peri                                       2   punct   _         _
\end{verbatim}
\end{list}
}
\medskip

\section{Middle Polish Linguistic Resources}
\label{sec:resources}
\subsection{KorBa}

KorBa \parencite{Gruszczynski2022KorBa} -- from Polish \textit{Korpus Barokowy} (`Baroque Corpus') -- is a~13.5-million-token corpus of~Polish texts from~1601–1772, compiled from~over seven hundred sources
and annotated morphosyntactically (lemmas, POS, features). It is searchable via~MTAS (Multi Tier Annotation Search; \cite{Brouwer-etal-2017}),
and provides parallel transliteration/transcription layers, structural and language markup, and rich metadata
(period, region, text type, genre) that enable stratified analyses.

The corpus includes diverse text types ranging from literary works (epic poetry, drama, lyric poetry) to non-literary materials (scientific-didactic texts, persuasive writings, factual literature, official documents, press releases) and biblical texts. Geographically, texts span the Polish-Lithuanian Commonwealth, with approximately 27\% of the corpus being of unknown origin. As shown in Figures~\ref{fig:korba-map} and \ref{fig:korba-sources}, the corpus maintains careful balance across regions and text types to ensure representativeness of Middle Polish.

\begin{figure}[H]
  \centering
  \includegraphics[width=\linewidth]{figs/korba_map.pdf}
  \caption{Geographical distribution of~texts in~the corpus displayed on~the map of~the Commonwealth after~the Union of~Lublin of~1569. Source: \textcite{Gruszczynski2022KorBa}, p. 315, CC~BY~4.0.}
  \label{fig:korba-map}
\end{figure}

\begin{figure}[H]
  \centering
  \includegraphics[width=\linewidth]{figs/korba_sources.pdf}
  \caption{Types of~texts in KorBa. Source: \textcite{Gruszczynski2022KorBa}, p. 316, CC~BY~4.0.}
  \label{fig:korba-sources}
\end{figure}


\subsection{MPDT}

The Middle Polish Dependency Treebank (MPDT) is a manually curated, syntactically annotated subset of the KorBa corpus, capturing key syntactic phenomena of 17th–18th-century Polish texts. The sentences are from the manually annotated part of KorBa, whose careful pre-processing provides reliable morphosyntactic annotation and balanced coverage across genres and periods. In its current form, MPDT represents the first systematic attempt at syntactic annotation of Middle Polish and therefore in the current version excludes poetry and sentences with Latin insertions, while limiting sentence length to 10–50 tokens, with the average sentence length being 23 tokens \parencite{Wieczorek2025TowardsMPDT}.


The annotation workflow consists of the following steps:
\begin{enumerate}[leftmargin=2em]
  \item \textbf{Automatic pre-annotation.} Two parsers trained on contemporary PDB data (MaltParser, COMBO) generate initial dependency analyses.
  \item \textbf{Manual correction.} Two linguist annotators independently revise parser outputs, leveraging complementary error profiles.
  \item \textbf{Adjudication.} Conflicting annotations are resolved by an adjudicator to produce a single gold-standard tree.
  \item \textbf{Formatting.} Final annotations are encoded in CoNLL-X
  \footnote{CoNLL-X is the predecessor of the CoNLL-U format \parencite{buchholz-marsi-2006-conll}} with KorBa’s extended tagset (e.g., dual number \texttt{Dual}).
\end{enumerate}


\paragraph{Corpus statistics}
\begin{itemize}[leftmargin=2em]
  \item Total sentences: 2 018  
  \item Total tokens: 47 273  
  \item Distinct POS tags: 45 
  \item Distinct dependency relations: 27 
  \item Non-projective edges: 3 748 across 879 sentences  
  \item Average sentence length: 23.43 tokens 
\end{itemize}

Figure~\ref{fig:mpdt-pos-frequency} presents the 20 most frequent MPDT POS tags, highlighting the prominence of nouns (\texttt{subst}: 11,374 occurrences), punctuation (\texttt{interp}: 7,971), adjectives (\texttt{adj}: 5,315), and prepositions (\texttt{prep}: 4,391).

\begin{figure}[H]
  \centering
  \includegraphics[width=0.8\textwidth]{./figs/pos_frequency.png}
  \caption{Top 20 MPDT POS tag frequencies}
  \label{fig:mpdt-pos-frequency}
\end{figure}

Figure~\ref{fig:mpdt-dep-frequency} shows the distribution of the top 20 dependency relation types. Adjuncts (\texttt{adjunct}: 13,276) and complements (\texttt{comp}: 8,539) are most common, followed by punctuation (\texttt{punct}: 6,896), coordination elements (\texttt{conjunct}: 6,071), and core arguments (\texttt{obj}: 3,423; \texttt{subj}: 2,286).

\begin{figure}[H]
  \centering
  \includegraphics[width=0.8\textwidth]{./figs/dep_frequency.png}
  \caption{Top 20 MPDT dependency relation base frequencies}
  \label{fig:mpdt-dep-frequency}
\end{figure}


\end{document}