%!TEX root = ../main.tex
\documentclass[../main.tex]{subfiles}
\begin{document}

This chapter presents the final, tangible outcome of the thesis: the initial public release of the \textbf{Middle Polish Dependency Treebank in Universal Dependencies format (MPDT-UD)}. Beyond reporting that the treebank passes the official validator, the chapter explains \emph{how} validation shaped the converter’s design: validator feedback drove targeted, logged corrections rather than ad-hoc edits, yielding a reproducible and auditable pipeline.

The chapter is structured as follows. Section~\ref{sec:val_methodology} introduces the validation workflow, including the iterative procedure, the validation dataset, and the official UD validator. Section~\ref{sec:val_results} reports the conformance results and outlines how remaining edge cases were handled. Section~\ref{sec:val_outcomes} presents the final MPDT-UD~1.0 treebank, summarizing its size, data split, licensing, and integration into the UD ecosystem.

\section{Validation Workflow and Formal Conformance}
\label{sec:val_methodology}


To fulfill research goal~\textbf{(R3)}—ensuring that the converted treebank is formally correct and compatible with standard UD tools—a strict validation workflow was adopted. The entire MPDT-UD corpus was checked using the \textbf{official Universal Dependencies validator script} \texttt{validate.py}.\footnote{The validator is part of the UD tools repository; available at \url{https://github.com/UniversalDependencies/tools/blob/master/validate.py}} The procedure was implemented as an iterative refinement cycle, applied to a fixed validation dataset and grounded in the behaviour of the official validator.

\subsection{Iterative Validation Procedure}
\label{sec:val_iterative}

The validation workflow was integrated into the development process as an iterative, data-driven refinement cycle that directly influenced the converter’s architecture. Achieving the final zero-error conformance required repeatedly executing the following steps:

\begin{enumerate}[leftmargin=1.5em]
  \item Running the full MPDT~$\rightarrow$~MPDT-UD conversion pipeline.
  \item Executing the \texttt{validate.py} script on the resulting \texttt{.conllu} file.
  \item Collecting and classifying validator messages to identify systematic error types.
  \item Implementing targeted adjustments in the relevant module (typically postconversion or label mapping) and repeating the cycle.
\end{enumerate}

Throughout development, validator feedback served as an objective convergence criterion: changes to the pipeline were accepted only if they reduced violations without introducing new ones, and every change was captured by the converter’s logging facilities for later audit.

\subsection{Validation Dataset}
\label{sec:val_corpus}

The validation was performed on the entire output of the conversion pipeline, i.e.\ the full MPDT available at the time of writing, converted to UD as MPDT-UD. All mentions in this chapter refer to this complete set.

Concretely, the validation dataset corresponds to the initial UD release of UD\_Polish-MPDT (version~2.17). It contains \textbf{2{,}018 sentences}. The material is drawn from the manually annotated portion of the KorBa corpus (as described in Section~\ref{sec:resources}) and thus inherits its genre balance and time-span.

Since Middle Polish is not a separate language branch within the UD project, the treebank was validated under the tagset and feature inventory of modern Polish (\texttt{pl}). This approach guarantees maximal compatibility with existing UD resources and ensures that the resulting data can be immediately processed by any standard UD-compliant parser or visualization tool.

\subsection{Official UD Validator}
\label{sec:val_validator}

The \texttt{validate.py} script performs a comprehensive, multi-layer verification of every sentence in a~CoNLL-U file. It is implemented in Python and relies on the official UD feature inventories and relation lists, performing checks in three main categories:

\begin{itemize}[leftmargin=1.5em]
  \item \textbf{Format compliance:} ensures that each sentence adheres to the CoNLL-U specification—exactly ten columns per token, valid multiword token ranges, correct comments, and consistent \texttt{SpaceAfter=No} annotation.
  \item \textbf{Morphological validity:} verifies that every \texttt{UPOS} tag, \texttt{XPOS} tag, and \texttt{FEATS} combination is legal according to UD~v2.17 inventories (e.g., that \texttt{Aspect} features only occur on verbal categories, and that \texttt{PronType} values correspond to pronouns or determiners).
  \item \textbf{Syntactic structure:} checks that the dependency tree is single-rooted, acyclic, and projective when required; that function words (e.g., tokens with relations \texttt{case}, \texttt{cc}, \texttt{mark}) do not have their own dependents; and that all \texttt{HEAD} indices refer to valid token IDs.
\end{itemize}

The validator also provides detailed diagnostic messages for every detected issue, grouped by error type and line number. This makes it suitable not only for final compliance testing but also for iterative debugging during development.

The script must be run with a language code (e.g., \texttt{--lang pl}) to check against treebank-specific feature and relation lists. It requires Python~3 and the \texttt{regex} and \texttt{udapi} modules. A typical invocation looked like:
\begin{verbatim}
python validate.py --lang pl MPDT-UD.conllu
\end{verbatim}


\section{Conformance Results}
\label{sec:val_results}

The final converted MPDT-UD treebank passes the official UD validator (\texttt{validate.py}) with \textbf{zero errors}. This 100\% conformance fulfills the principal technical objective of the thesis~(\textbf{R3}) and confirms that the resource is formally compatible with the Universal Dependencies standard.

This outcome is the result of the iterative, data-driven refinement cycle described in Section~\ref{sec:val_iterative}. Validator feedback was treated as an objective convergence criterion: pipeline changes were accepted only when they reduced the number of violations without introducing new ones, and all modifications were recorded by the converter’s logging facilities for later audit.

The emphasis throughout was on \emph{principle-driven} cleanups aligned with UD guidance rather than ad hoc edits—for example, enforcing that function words do not bear dependents, attaching punctuation to appropriate content heads, and canonicalizing order-sensitive multiword relations. In this way, the final state of the converter encodes the corrections required to reach full conformance, rather than relying on one-off manual adjustments to the released data.

\subsection{Handling Idiosyncratic Edge Cases}
\label{sec:val_results_idiosyncratic}

Most validation issues were solvable with generalized, context-sensitive rules that can be applied uniformly across the corpus. A small residual set of idiosyncratic cases—rare constructions not easily captured by broad heuristics—were addressed pragmatically to achieve full conformance on the released dataset. These interventions are minimal, logged, and isolated, preserving reproducibility without overfitting the pipeline to particular sentences.

Conversion decisions for \textbf{POS} and \textbf{FEATS} are documented systematically in the accompanying specification.\footnote{See \url{https://github.com/kvmilos/MPDT-to-UD-converter/blob/main/MPDT.md}} This document provides a stable reference for future extensions of the pipeline and for users who wish to understand or reuse the morphosyntactic mapping outside the present project.


\section{Outcomes: the MPDT-UD 1.0 Treebank}
\label{sec:val_outcomes}

The validated conversion pipeline yields the first public release of the Middle Polish Dependency Treebank in Universal Dependencies format (\textbf{MPDT-UD~1.0}). This section summarizes the size and internal structure of the treebank and briefly documents its integration into the Universal Dependencies ecosystem, including licensing and availability.

\subsection{Corpus Size and Data Split}
\label{sec:val_outcomes_stats}

The MPDT-UD~1.0 treebank contains \textbf{2{,}018 sentences}, \textbf{46{,}670 surface tokens}, and \textbf{47{,}273 syntactic words}. The difference between tokens and syntactic words reflects the use of multiword tokens in the CoNLL-U representation. The corpus includes \textbf{564 multiword tokens} (across \textbf{359} distinct types), with an average of 2.07 syntactic words per multiword token. In addition, \textbf{8{,}217 tokens} (18\%) are not followed by a space, which is a direct consequence of historical orthographic conventions such as clitic fusion and non-standard punctuation spacing.\footnote{Summary statistics and tokenization details are publicly available via the UD\_Polish-MPDT treebank page: \url{https://universaldependencies.org/treebanks/pl_mpdt/index.html}.}

In line with the UD guidelines for treebanks in the 20K–110K word range, approximately 10K words were allocated to the test set, about 10\% of the remaining data to the development set, and the rest to the training set. Sentences were randomly shuffled (using seed~42) and then assigned to the three subsets based on token-count quotas. The resulting split is:

\begin{itemize}[leftmargin=2em]
  \item \textbf{Training set:} 1{,}433 sentences, 33{,}520 tokens.
  \item \textbf{Development set:} 162 sentences, 3{,}748 tokens.
  \item \textbf{Test set:} 423 sentences, 10{,}005 tokens.
\end{itemize}

The treebank does not preserve document boundaries or genre labels at the sentence level; sentences have been shuffled and redistributed purely for the purpose of forming these three subsets. At the corpus level, the metadata assign four broad genres to the resource as a whole: \textsl{nonfiction}, \textsl{bible}, \textsl{legal}, and \textsl{fiction}.

\subsection{Release, Licensing, and UD Integration}
\label{sec:val_outcomes_release}

MPDT-UD~1.0 is distributed within the Universal Dependencies project under the repository name \texttt{UD\_Polish-MPDT}. It was first released as part of the \textbf{UD~v2.17} release \parencite{11234/1-6036}. The data are available both through the UD GitHub organization and via the standard UD release packages, alongside other Polish resources.

The treebank is released under the \textbf{Creative Commons Attribution–ShareAlike 4.0 (CC~BY–SA~4.0)} license.\footnote{License text and a plain-language summary are available at \url{https://creativecommons.org/licenses/by-sa/4.0/}.} In practical terms:

\begin{itemize}[leftmargin=2em]
  \item \textbf{You may} share and adapt the data, including for commercial purposes.
  \item \textbf{You must} give appropriate credit, provide a link to the license, and indicate if changes were made.
  \item \textbf{ShareAlike:} if you remix, transform, or build upon the material, you must distribute your contributions under the same license as the original.
  \item \textbf{No additional restrictions:} you may not apply legal terms or technological measures that legally restrict others from doing anything the license permits.
\end{itemize}

These conditions align with the open-data practices of the UD project and make MPDT-UD~1.0 immediately reusable for historical syntax research, diachronic NLP experiments, and cross-linguistic studies that rely on UD-conformant training and evaluation data.


\end{document}