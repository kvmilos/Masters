%!TEX root = ../main.tex
\documentclass[../main.tex]{subfiles}
\begin{document}

\section{Motivation}

Natural-language preprocessing tools and comparative treebank research have standardized around~Universal Dependencies (UD),
which enables typologically informed analyses and cross-lingual transfer \parencite{nivre-etal-2020-universal}.
For~Polish texts from the 17th and 18th centuries, however, key resources remain outside~UD:
texts in~the KorBa corpus \parencite{Gruszczynski2022KorBa} and the emerging Middle Polish Dependency Treebank
(MPDT) are annotated in~a~Polish-specific scheme \parencite{Wieczorek2025TowardsMPDT}. KorBa is a corpus of historical Polish texts, while MPDT adds a syntactic dependency layer to selected portions of this corpus. However, these resources being annotated in a different format creates challenges
for interoperability with~UD-based tools and limits straightforward comparative studies with~other languages.

A natural solution is to convert these resources to the UD format. From~an engineering perspective, however, a~faithful, auditable conversion is non-trivial:
historical orthography, abbreviations, clitic mobility, 
numeral complexes, and multiword conjunctions/prepositions interact with~head rules and label inventories.
Prior conversion experience for~contemporary Polish offers valuable guidance
\parencite{wroblewska-2018-extended, wroblewska-2020-towards},
yet historical data introduce additional phenomena that require explicit, rule-based handling and transparent traceability.

As~\textcite{Wieczorek2025TowardsMPDT} notes, MPDT's current format is~well-suited to~comparative
studies with~contemporary Polish syntax; at~the same time, she highlights the advantages of~moving to~UD
for~cross-linguistic comparability, wider intelligibility, and representational options such as~enhanced dependencies
for~shared dependents and shared governors in~coordination—even if~some information may be lost in~conversion.

This thesis operationalizes that rationale by~delivering a~documented, UD-oriented converter for~MPDT and preparing
the current version of MPDT-UD suitable for~validation and downstream use.
The intended users include historical linguists needing UD-compatible data and~NLP practitioners interested in~diachronic Polish or~cross-lingual experiments.


\section{Objectives}

The thesis pursues the following research goals:

\begin{enumerate}[label=\textbf{(R\arabic*)}, leftmargin=2em]
  \item \textbf{Design a~UD-oriented conversion strategy for~MPDT.}
        Specify mapping principles that respect Middle Polish specifics while aligning with~UD guidelines.
  \item \textbf{Implement an auditable conversion pipeline.}
        Provide modular components for~morphosyntax mapping and dependency restructuring, with~token-level logging.
  \item \textbf{Ensure UD conformance and evaluability.}
        Produce output that passes the official UD validator (on all levels) and supports downstream analysis.
  \item \textbf{Document decisions.}
        Record non-obvious mapping choices and edge-case policies to~enable maintenance and reuse.
\end{enumerate}


\section{Contributions}

This project delivers concrete, reusable artifacts:

\begin{enumerate}[label=\textbf{(C\arabic*)}, leftmargin=2em]
  \item \textbf{A~rule-based MPDT~$\rightarrow$~MPDT-UD converter.}
        A~modular pipeline with~fine-grained logging, selectively adapting ideas from \textcite{wroblewska-2018-extended} while
        targeting Middle Polish phenomena. The code will be released in~a~public repository under~an open-source
        license, together with~this thesis, which documents the design and implementation.
  \item \textbf{An initial public release of~MPDT-UD.}
        A~set of~MPDT (2018 sentences at~the time of~writing) converted automatically and validated with~the
        official UD validator.
\end{enumerate}


\section{Structure of the Thesis}

\begin{itemize}[leftmargin=1.5em]
  \item \textbf{Chapter~2: Background.} Introduces dependency grammar and the Polish Dependency Bank (PDB) annotation scheme; outlines the Universal Dependencies (UD) framework, including its layers and relation inventories; and presents the key source resources—KorBa and MPDT—that the conversion operates on.
  \item \textbf{Chapter~3: Linguistic Features of Middle Polish.} Describes linguistic properties of Middle Polish relevant to conversion: orthography and punctuation practices; characteristic morphological categories (\texttt{adjb}, \texttt{ppasb}, \texttt{ppraet}, dual number); evolving masculine gender distinctions; the connective \emph{jako} in its comparative and role uses; and syntactic features such as non-projectivity and predicate ellipsis.
  \item \textbf{Chapter~4: Conversion Design and Implementation.} Details the custom Python pipeline: its modular architecture, core \texttt{Sentence} and \texttt{Token} classes, and auditable logging system. It explains the two-phase process—(1) rule-based morphosyntactic mapping of POS tags and features, and (2) dependency restructuring—together with label mapping and post-processing that ensure full UD conformance.
  \item \textbf{Chapter~5: Validation and Outcomes.} Introduces the validation workflow based on the official UD validator, including the iterative procedure and validation dataset; reports the conformance results and treatment of remaining edge cases; and presents the final MPDT-UD~1.0 treebank, summarizing its size, data split, licensing, and integration into the UD ecosystem.
  \item \textbf{Chapter~6: Applications and Cognitive Science Perspective.} Positions the converter and MPDT-UD as reusable infrastructure, sketches applications in historical syntax and diachronic NLP, and relates the resource to questions about processing constraints and category change over time, concluding with proposals for future extensions of the pipeline and treebank.
\end{itemize}

\end{document}